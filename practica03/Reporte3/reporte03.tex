
\documentclass[10pt,journal,compsoc]{IEEEtran}\usepackage[T1]{fontenc}                              % Indicamos la codificacion de las fuentes.
\usepackage[utf8x]{inputenc}                          % Definimos la codificacion.
\usepackage{lmodern}                                  % Para poder usar acentos.
\usepackage[spanish,es-nodecimaldot]{babel}           % Usaremos idioma español y punto decimal.
\usepackage{listings}                                 % Para código
\usepackage[numbered,framed]{matlab-prettifier}       % Para código M
\usepackage{graphicx}

\ifCLASSOPTIONcompsoc
  
  \usepackage[nocompress]{cite}
\else
  
  \usepackage{cite}
\fi

\ifCLASSINFOpdf

\else
 
\fi

\usepackage{amsmath}

\interdisplaylinepenalty=2500

\usepackage{algorithmic}

\newcommand\MYhyperrefoptions{bookmarks=true,bookmarksnumbered=true,
pdfpagemode={UseOutlines},plainpages=false,pdfpagelabels=true,
colorlinks=true,linkcolor={black},citecolor={black},urlcolor={black},
pdftitle={Bare Demo of IEEEtran.cls for Computer Society Journals},%<!CHANGE!
pdfsubject={Typesetting},%<!CHANGE!
pdfauthor={Michael D. Shell},%<!CHANGE!
pdfkeywords={Computer Society, IEEEtran, journal, LaTeX, paper,
             template}}%<^!CHANGE!

\hyphenation{op-tical net-works semi-conduc-tor}


\begin{document}
%
% paper title
% Titles are generally capitalized except for words such as a, an, and, as,
% at, but, by, for, in, nor, of, on, or, the, to and up, which are usually
% not capitalized unless they are the first or last word of the title.
% Linebreaks \\ can be used within to get better formatting as desired.
% Do not put math or special symbols in the title.
\title{RECONOCIMIENTO POR TEXTURA }
%
%
% author names and IEEE memberships
% note positions of commas and nonbreaking spaces ( ~ ) LaTeX will not break
% a structure at a ~ so this keeps an author's name from being broken across
% two lines.
% use \thanks{} to gain access to the first footnote area
% a separate \thanks must be used for each paragraph as LaTeX2e's \thanks
% was not built to handle multiple paragraphs
%
%
%\IEEEcompsocitemizethanks is a special \thanks that produces the bulleted
% lists the Computer Society journals use for "first footnote" author
% affiliations. Use \IEEEcompsocthanksitem which works much like \item
% for each affiliation group. When not in compsoc mode,
% \IEEEcompsocitemizethanks becomes like \thanks and
% \IEEEcompsocthanksitem becomes a line break with idention. This
% facilitates dual compilation, although admittedly the differences in the
% desired content of \author between the different types of papers makes a
% one-size-fits-all approach a daunting prospect. For instance, compsoc 
% journal papers have the author affiliations above the "Manuscript
% received ..."  text while in non-compsoc journals this is reversed. Sigh.

\author{Paul Sebastian~Aguilar Enriquez,
	Carlos Ignacio~Padilla Herrera
	y ~Simón Eduardo~Ramírez Ancona% <-this % stops a space
%\IEEEcompsocitemizethanks{
%  \IEEEcompsocthanksitem Pie 
 % \IEEEcompsocthanksitem Pie
%}% <-this % stops a space
%\thanks{Referencia de pie}
}

% note the % following the last \IEEEmembership and also \thanks - 
% these prevent an unwanted space from occurring between the last author name
% and the end of the author line. i.e., if you had this:
% 
% \author{....lastname \thanks{...} \thanks{...} }
%                     ^------------^------------^----Do not want these spaces!
%
% a space would be appended to the last name and could cause every name on that
% line to be shifted left slightly. This is one of those "LaTeX things". For
% instance, "\textbf{A} \textbf{B}" will typeset as "A B" not "AB". To get
% "AB" then you have to do: "\textbf{A}\textbf{B}"
% \thanks is no different in this regard, so shield the last } of each \thanks
% that ends a line with a % and do not let a space in before the next \thanks.
% Spaces after \IEEEmembership other than the last one are OK (and needed) as
% you are supposed to have spaces between the names. For what it is worth,
% this is a minor point as most people would not even notice if the said evil
% space somehow managed to creep in.



% The paper headers
\markboth{Reconocimiento de patrones, FI-UNAM, 2020-1}%
{Shell \MakeLowercase{\textit{et al.}}: Bare Advanced Demo of IEEEtran.cls for IEEE Computer Society Journals}
% The only time the second header will appear is for the odd numbered pages
% after the title page when using the twoside option.


% for Computer Society papers, we must declare the abstract and index terms
% PRIOR to the title within the \IEEEtitleabstractindextext IEEEtran
% command as these need to go into the title area created by \maketitle.
% As a general rule, do not put math, special symbols or citations
% in the abstract or keywords.
\IEEEtitleabstractindextext{%
\begin{abstract}
Este documento presenta la implementación de un Reconocedor de texturas para imágenes de brodatz utilizando MATLAB.
Se muestran imágenes del proceso, así como del resultado final. 
Se da un breve introducción de image retrieval, análisis de texturas, métodos de validación y KNN.
Se muestra el código fuente del clasificador implementado.
\end{abstract}

% Note that keywords are not normally used for peerreview papers.
\begin{IEEEkeywords}
Pattern Recognition, Image retrieval, KNN, PCA, Texture analysis, MATLAB.
\end{IEEEkeywords}


% make the title area
\maketitle


% To allow for easy dual compilation without having to reenter the
% abstract/keywords data, the \IEEEtitleabstractindextext text will
% not be used in maketitle, but will appear (i.e., to be "transported")
% here as \IEEEdisplaynontitleabstractindextext when compsoc mode
% is not selected <OR> if conference mode is selected - because compsoc
% conference papers position the abstract like regular (non-compsoc)
% papers do!
\IEEEdisplaynontitleabstractindextext
% \IEEEdisplaynontitleabstractindextext has no effect when using
% compsoc under a non-conference mode.

%
% For peerreview papers, this IEEEtran command inserts a page break and
% creates the second title. It will be ignored for other modes.
\IEEEpeerreviewmaketitle

\section{Objetivo}

El alumno:

\begin{enumerate}
  \item aprenderá el concepto de "image retrieval" basado en análisis de texturas.
  \item entenderá cuando y cómo utilizar clasificadores como KNN, LDA (Fisher) o máquinas de soporte vectorial (SVM).
 % \item Entenderá cómo aumentar las características de contraste y perfilado que apoyen a una mejor medición.
\end{enumerate}



\section{Introducción}
 
% Here we have the typical use of a "T" for an initial drop letter
% and "HIS" in caps to complete the first word.
\IEEEPARstart{R}ealice una investigación sobre image retrieval, análisis de texturas, métodos de validación y clasificadores
KNN, LDA y máquinas de soporte vectorial.




 
\hfill

\section{Desarrollo}

\begin{enumerate}
  \item Generamos un sistema de recuperación de imágenes mediante un proceso de reconocimiento de patrones.
  \item Usamos 5 imágenes para el proceso.
  \item Subdividimos la imágen en varias subimagenes, guardando 3 de las mismas para el proceso de recuperación de carácteristicas
  que entrega la matriz de Haralick o gray level cocurrence matrix (GLCM).
  \item Obtuvimos la entropía, energía y generamos el vector de carácteristicas.
  \item Aplicamos un clasificador con los vectores de datos obtenidos.
  \item Programamos un clasificador basado en la distancia mínima entre vectores.
  \item Usamos LDA, Bayes, KNN y SVM.
  \item Realizamos una comparación entre clasificadores
\end{enumerate}


%\graphicspath{ {C:\Users\Enigm\clase-reconocimiento-patrones-2020-1\practica01\ReporteImagenes/RP_1} }
%Path in Unix-like (Linux, Mac OS) format
%\graphicspath{ {/home/user/images/} }
%\includegraphics[scale=1.5]{lion-logo} 

En las figuras mostradas se puede observar el resultado de aplicar el recorte para que solo quede la información del espacio vacio y de la aplicación  de un filtro para homogeneizar el color de cada una de las clases.
Este preprocesamiento de las imágenes se realiza para reducir el ruido y homogeneizar las imagenes y poder obtener un mejor resultado.
A continuación se modificó el tamaño y propiedades de las imagenes para homogeneizarlas:

\begin{lstlisting}[language=Matlab, basicstyle=\small]
% Codigo en MATLAB de la practica 3

\end{lstlisting}

\section{Resultados}

%Los resultados deberán presentarse con los cálculos respectivos.

En las figuras se puede observar la comparación entre las imágenes a las que se les aplico el filtro gaussiano y las imágenes que entregó nuestro clasificador .
En general se puede decir que el clasificador tiene un comportamiento bastante aceptable ya que la mayoría de los pixeles están bien clasificados en la clase correspondiente del cúmulo de estrellas, sin embargo, puede ser mejorado, ya que algunas zonas no están bien clasificadas, sobre todo los límites de esta región, y se podría mejorar la detección de los mismos y una mejor diferenciación entre las tres regiones.

\graphicspath{{ReporteImagenes/}}
%Se empezó con la imagen original


%\begin{figure}[!h]
%	\caption{Imagen original}
%	\centering
%	\includegraphics*[scale=0.2]{im1}
%	\centering
%\end{figure}


%\begin{figure}[!h]
%	\caption{Imagen original}
%	\centering
%	\includegraphics*[scale=0.2]{im2}
%	\centering
%\end{figure}


%\begin{figure}[!h]
%	\caption{Imagen original}
%	\centering
%	\includegraphics*[scale=0.2]{im3}
%	\centering
%\end{figure}


%\begin{figure}[!h]
%	\caption{Histograma de la imagen filtrada y ajustada}
%	\centering
%	\includegraphics*[scale=0.2]{im4}
%	\centering
%\end{figure}


%\begin{figure}[!h]
%	\caption{Imagen final recortada}
%	\centering
%	\includegraphics*[scale=0.4]{RP_6}
%	\centering
	
%\end{figure}


%\begin{figure}[!h]
%	\caption{Imagen final recortada}
%	\centering
%		\includegraphics*[scale=0.4]{RP_7}
%	\centering
% \end{figure}

% \begin{figure}[!h]
%	\caption{Imagen final recortada}
%	\centering
%	\includegraphics*[scale=0.4]{RP_8}
%	\centering
% \end{figure}




\section{Código fuente}

\begin{lstlisting}

  Insertar codigo en matlab Aquí


\end{lstlisting}
\section{Conclusiones}

El clasificador bayesiano es una herramienta ampliamente usada y es bastante sencilla de aplicar, ya que solo emplea conceptos de probabilidad, además devuelve resultados bastante aceptables para la clasificación de imagenes.
Los resultados obtenidos con las imagenes fueron bastante aceptables, ya que la mayoría de los pixeles están bien clasificados. Con algunos pequeños retoques es probable que la implementación sea capaz de tener un mejor desempeño.


%\begin{figure}[!t]
%\centering
%\includegraphics[width=2.5in]{myfigure}
% where an .eps filename suffix will be assumed under latex, 
% and a .pdf suffix will be assumed for pdflatex; or what has been declared
% via \DeclareGraphicsExtensions.
%\caption{Simulation results for the network.}
%\label{fig_sim}
%\end{figure}


%\begin{figure*}[!t]
%\centering
%\subfloat[Case I]{\includegraphics[width=2.5in]{box}%
%\label{fig_first_case}}
%\hfil
%\subfloat[Case II]{\includegraphics[width=2.5in]{box}%
%\label{fig_second_case}}
%\caption{Simulation results for the network.}
%\label{fig_sim}
%\end{figure*}


%\begin{table}[!t]
%% increase table row spacing, adjust to taste
%\renewcommand{\arraystretch}{1.3}
% if using array.sty, it might be a good idea to tweak the value of
% \extrarowheight as needed to properly center the text within the cells
%\caption{An Example of a Table}
%\label{table_example}
%\centering
%% Some packages, such as MDW tools, offer better commands for making tables
%% than the plain LaTeX2e tabular which is used here.
%\begin{tabular}{|c||c|}
%\hline
%One & Two\\
%\hline
%Three & Four\\
%\hline
%\end{tabular}
%\end{table}

% references section

% can use a bibliography generated by BibTeX as a .bbl file
% BibTeX documentation can be easily obtained at:
% http://mirror.ctan.org/biblio/bibtex/contrib/doc/
% The IEEEtran BibTeX style support page is at:
% http://www.michaelshell.org/tex/ieeetran/bibtex/
%\bibliographystyle{IEEEtran}
% argument is your BibTeX string definitions and bibliography database(s)
%\bibliography{IEEEabrv,../bib/paper}
%
% <OR> manually copy in the resultant .bbl file
% set second argument of \begin to the number of references
% (used to reserve space for the reference number labels box)
\begin{thebibliography}{1}

\bibitem{IEEEhowto:kopka}
W.~Pratt, \emph{Digital Image Processing}, John Wiley \& Sons Inc, 2001.

\bibitem{IEEEhowto:kopka}
Gonzales Woods, \emph{Digital Image Processing}, 2004.

\end{thebibliography}


% that's all folks
\end{document}
